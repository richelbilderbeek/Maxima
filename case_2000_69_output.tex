\documentclass{article}

\usepackage{listings}
\usepackage{graphicx}

\title{Case 2000 page 69 exercise}
\author{Richel Bilderbeek}
\date{\today}

\begin{document}

\maketitle

\begin{abstract}
This article is created within the CAS program Maxima
and shows how to the exercise of \cite{case2000} at page 69.
\end{abstract}

\section{Introduction}

\LaTeX~is commonly used for writing publishable scientific articles\cite{gaudeul2006}.
Algebraic manipulations can be done by a CAS, for example Maxima, Maple or Mathematica.
Maxima is the only free and open-source program, and it is the oldest free and open-source computer algebra system, with development started in 1967 (as Macsyma) or 1982 (as MAXIMA).
This article is an example of writing a \LaTeX~ article within Maxima

\section{Exercise}

\begin{table}[here]
  \centering
  \begin{tabular}{ | r | l | }
    \hline
    symbol & description \\
    \hline
    $I$ & Indentity matrix \\
    $L$ & Leslie matrix \\
    $M$ & Leslie matrix with $\lambda$ subtracted at diagonal \\
    $x$ & population density vector \\
    $Z$ & Vector filled with zeroes \\
    $\lambda$ & population growth rate \\
    \hline
  \end{tabular}
  \caption{Definitions}
  \label{table:table_definition}
\end{table}

(for definitions see table \ref{table:table_definition} on page \pageref{table:table_definition}).

The equation to solve, equation 3.21 is:
\begin{equation}
x\,L=x\,\lambda\end{equation}
The Leslie matrix, L,  given is:

$$\pmatrix{1.0&1.0\cr 0.8&0.8\cr }$$

The simplifies equation 3.21 to:

$$\pmatrix{1.0\,x&1.0\,x\cr 0.8\,x&0.8\,x\cr }=x\,\lambda$$

Solving equation 3.21 can be done with equation 3.24:

$$det(L - \lambda*I) = Z$$

Where I is the identity matrix:
$$\pmatrix{1&0\cr 0&1\cr }$$
And Z is the vector of zeroes:
$$\pmatrix{0\cr 0\cr }$$

This simplifies equation 3.24 to:

$${\it det}\left(\pmatrix{1.0-\lambda&1.0\cr 0.8&0.8-\lambda\cr }
 \right)=\pmatrix{0\cr 0\cr }$$


The determinant of that matrix ($M$), is:
$$\left(0.8-\lambda\right)\,\left(1.0-\lambda\right)-0.8$$


Solving $M=0$, the $\lambda$s found are:

$$\left[ \lambda={{9}\over{5}} , \lambda=0 \right] $$

There is one stable population structure, $\lambda=0$, which is
denotes an extinct population.
Here I focus on the more interesting value,
where $\lambda=9/5$.
This lambda is called the dominant eigenvalue, which equals the ultimate population growth.

Results in M:
$$\pmatrix{-0.8&1.0\cr 0.8&-1.0\cr }$$

Solving 3.23: $M * x = Z$:

Create a population vector, $x$, with $1.0$ as an initial value
(it will be rescaled later):

$$\pmatrix{1.0\cr {\it x_2}\cr }$$
Now $M * x$ simplifies to:
$$\pmatrix{1.0\,{\it x_2}-0.8\cr 0.8-1.0\,{\it x_2}\cr }$$
Solving this, 3.23: $M * x = Z$, there are two equations that can be solved:
$$\pmatrix{1.0\,{\it x_2}-0.8\cr 0.8-1.0\,{\it x_2}\cr }=\pmatrix{0
 \cr 0\cr }$$
Solving the upper, results in $x2=4/5$.
This results in an $x$ of:
$$\pmatrix{1.0\cr {{4}\over{5}}\cr }$$
$x$ must be rescaled so that its sum equals $1.0$.
$x$ its current sum is $1.8$, so dividing all elements by it, results in an $x$ of:
$$\pmatrix{0.55555555555555\cr 0.44444444444444\cr }$$
\section{Conclusion}

For this Leslie matrix:
$$\pmatrix{1.0&1.0\cr 0.8&0.8\cr }$$

The dominant eigenvalue, $\lambda$, is:
\\\\
9/5\\

The stable population size distribution is:

$$\pmatrix{0.55555555555555\cr 0.44444444444444\cr }$$

\section{Discussion}

Writing \LaTeX~within Maxima can be done, but it is a bit cumbersome:
Maxima does not know \LaTeX~syntax and just creates contextless strings,
which might not be compilable by \LaTeX.
However, because the script does create a .tex file,
this file can be inspected easily with a \LaTeX~tool like texmaker.

\begin{thebibliography}{9}

\bibitem{case2000}
  Case, Ted J.
  2000
  An illustrated guide to theoretical ecology.

\bibitem{gaudeul2006}
  Gaudeul, A.
  2006
  Do Open Source Developers Respond to Competition?: The (La)TeX Case Study.
  Available at SSRN: http://ssrn.com/abstract=908946 or http://dx.doi.org/10.2139/ssrn.908946

\end{thebibliography}

\appendix

\section{Script file}

\lstinputlisting[language=C++,showstringspaces=false,breaklines=true,frame=single]{case_2000_69.sh}

\section{Maxima file}

\lstinputlisting[language=C++,showstringspaces=false,breaklines=true,frame=single]{case_2000_69.txt}

\section{\LaTeX~file}

\lstinputlisting[language=tex,showstringspaces=false,breaklines=true,frame=single]{case_2000_69_output.tex}

\end{document}

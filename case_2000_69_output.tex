\documentclass{article}

\usepackage{listings}
\usepackage{graphicx}

\title{Case 2000 page 69 exercise}
\author{Richel Bilderbeek}
\date{\today}

\begin{document}

\maketitle

\begin{abstract}
This article is created within the CAS program Maxima
and shows how to the exercise of \cite{case2000} at page 69.
\end{abstract}

\section{Introduction}

\LaTeX~is commonly used for writing publishable scientific articles\cite{gaudeul2006}.
Algebraic manipulations can be done by a CAS, for example Maxima, Maple or Mathematica.
Maxima is the only free and open-source program, and it is the oldest free and open-source computer algebra system, with development started in 1967 (as Macsyma) or 1982 (as MAXIMA).
This article is an example of writing a \LaTeX~ article within Maxima

\section{Exercise}

\begin{table}[here]
  \centering
  \begin{tabular}{ | r | l | }
    \hline
    symbol & description \\
    \hline
    $I$ & Indentity matrix \\
    $L$ & Leslie matrix \\
    $M$ & Leslie matrix with $\lambda$ subtracted at diagonal \\
    $x$ & population density vector \\
    $Z$ & Vector filled with zeroes \\
    $\lambda$ & population growth rate \\
    \hline
  \end{tabular}
  \caption{Definitions}
  \label{table:table_definition}
\end{table}

The equations to solve are:
\begin{equation}
x\,L=x\,\lambda\label{eq:eq3_21}
\end{equation}
\begin{equation}
x\,M=Z\label{eq:eq3_23}
\end{equation}

Equation \ref{eq:eq3_21} equals equation 3.21 of \cite{case2000}
Equation \ref{eq:eq3_23} equals equation 3.23 of \cite{case2000}
(for definitions see table \ref{table:table_definition} on page \pageref{table:table_definition}).
The Leslie matrix, L,  given in this exercise is:

\begin{equation}
L=\pmatrix{1.0&1.0\cr 0.8&0.8\cr }\end{equation}

The Leslie matrix, L, used:

\begin{equation}
L=\pmatrix{0.1&2.0\cr 0.1&0.8\cr }\end{equation}


Note that this matrix has 2 age classes.
The simplifies equation \ref{eq:eq3_21} to:

\begin{equation}
\pmatrix{0.1\,x&2.0\,x\cr 0.1\,x&0.8\,x\cr }=x\,\lambda\end{equation}

Solving equation \ref{eq:eq3_21} can be done with equation \ref{eq:eq3_24}:

\begin{equation}
det(L - \lambda*I) = Z
\label{eq:eq3_24}
\end{equation}

Equation \ref{eq:eq3_24} equals equation 3.24 of \cite{case2000},
where I is the identity matrix:
\begin{equation}
I=\pmatrix{1&0\cr 0&1\cr }\end{equation}
And Z is the vector of zeroes:
\begin{equation}
Z=\pmatrix{0\cr 0\cr }\end{equation}

This simplifies equation \ref{eq:eq3_24} to:

\begin{equation}
{\it det}\left(\pmatrix{0.1-\lambda&2.0\cr 0.1&0.8-\lambda\cr }\right)=\pmatrix{0\cr 0\cr }\end{equation}


The determinant of that matrix ($M$), is:
\begin{equation}
det(\pmatrix{0.1-\lambda&2.0\cr 0.1&0.8-\lambda\cr })=\left(0.1-\lambda\right)\,\left(0.8-\lambda\right)-0.2\end{equation}


Solving $M=0$, the $\lambda$s found are:

\begin{equation}
\left[ \lambda=-{{\sqrt{129}-9}\over{20}} , \lambda={{\sqrt{129}+9}\over{20}} \right] \end{equation}

There is one stable population structure, $\lambda=0$, which is
denotes an extinct population.
Here I focus on the more interesting value,
where $\lambda=-(sqrt(129)-9)/20$.
This lambda is called the dominant eigenvalue, which equals the ultimate population growth.

Put $\lambda$ in M, this results in:
\begin{equation}
M=\pmatrix{{{\sqrt{129}-9}\over{20}}+0.1&2.0\cr 0.1&{{\sqrt{129}-9}\over{20}}+0.8\cr }\end{equation}

Now we can solve the $x$ of equation \ref{eq:eq3_23} (equals 3.23 from \cite{case2000}),
which was this:

\begin{equation}
x\,M=Z\end{equation}

This equation is unsolvable, unless we assign a value to an element of $x$.
Here, I put $1.0$ as the initial value of $x$ its first element.
(it will be rescaled later):

\begin{equation}
x = \pmatrix{1.0\cr {\it x_2}\cr }\end{equation}
Putting this $x$ in equation \ref{eq:eq3_23}:
\begin{equation}
M * x = \pmatrix{2.0\,{\it x_2}+1.0\,\left({{\sqrt{129}-9}\over{20}}+0.1\right)\cr \left({{\sqrt{129}-9}\over{20}}+0.8\right)\,{\it x_2}+0.1\cr }= \pmatrix{0\cr 0\cr }\end{equation}
As our matrix has two rows, there are two equations that can be solved:
Solving the upper, results in $x2=-(sqrt(129)-7)/40$.
This results in an $x$ of:
\begin{equation}
x = \pmatrix{1.0\cr -{{\sqrt{129}-7}\over{40}}\cr }\end{equation}
$x$ must be rescaled so that its sum equals $1.0$.
$x$ its current sum is $1.0-(sqrt(129)-7)/40$, so dividing all elements by it, results in an $x$ of:
\begin{equation}
x = \pmatrix{{{1.0}\over{1.0-{{\sqrt{129}-7}\over{40}}}}\cr -{{\sqrt{129}-7}\over{40\,\left(1.0-{{\sqrt{129}-7}\over{40}}\right)}}\cr }\end{equation}
\section{Conclusion}

For this Leslie matrix:
\begin{equation}
\pmatrix{0.1&2.0\cr 0.1&0.8\cr }\end{equation}

The dominant eigenvalue, $\lambda$, is:
\\\\
\begin{equation}
\lambda = -(sqrt(129)-9)/20\end{equation}
\\

The stable population size distribution is:

\begin{equation}
x = \pmatrix{{{1.0}\over{1.0-{{\sqrt{129}-7}\over{40}}}}\cr -{{\sqrt{129}-7}\over{40\,\left(1.0-{{\sqrt{129}-7}\over{40}}\right)}}\cr }\end{equation}

\section{Discussion}

Writing \LaTeX~within Maxima can be done, but it is a bit cumbersome:
Maxima does not know \LaTeX~syntax and just creates contextless strings,
which might not be compilable by \LaTeX.
However, because the script does create a .tex file,
this file can be inspected easily with a \LaTeX~tool like texmaker.

\begin{thebibliography}{9}

\bibitem{case2000}
  Case, Ted J.
  2000
  An illustrated guide to theoretical ecology.

\bibitem{gaudeul2006}
  Gaudeul, A.
  2006
  Do Open Source Developers Respond to Competition?: The (La)TeX Case Study.
  Available at SSRN: http://ssrn.com/abstract=908946 or http://dx.doi.org/10.2139/ssrn.908946

\end{thebibliography}

\appendix

\section{Script file}

\lstinputlisting[language=C++,showstringspaces=false,breaklines=true,frame=single]{case_2000_69.sh}

\section{Maxima file}

\lstinputlisting[language=C++,showstringspaces=false,breaklines=true,frame=single]{case_2000_69.txt}

\section{\LaTeX~file}

\lstinputlisting[language=tex,showstringspaces=false,breaklines=true,frame=single]{case_2000_69_output.tex}

\end{document}
